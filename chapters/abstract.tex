\newpage

\begin{center}
\textbf{Replicating DeepMind StarCraft II Reinforcement Learning Benchmark with Actor-Critic Methods}\\
Bachelor’s thesis\\
Roman Ring\\
\end{center}

\noindent 
\textbf{Abstract.} Reinforcement Learning (RL) is a subfield of Artificial Intelligence (AI) that deals with agents navigating in an environment with the goal of maximizing total reward. Games are good environments to test RL algorithms as they have simple rules and clear reward signals. Theoretical part of this thesis explores some of the popular classical and modern RL approaches, which include the use of Artificial Neural Network (ANN) as a function approximator inside AI agent. In practical part of the thesis we implement Advantage Actor-Critic RL algorithm and replicate ANN based agent described in~\cite{Vinyals2017}. We reproduce the state-of-the-art results in a modern video game StarCraft II, a game that is considered the next milestone in AI after the fall of chess and Go.
\\\\
\textbf{Keywords:} reinforcement learning, artificial neural networks 
\\\\
\textbf{CERCS research specialisation:} P176 Artificial intelligence
\\\\
\begin{center}
\textbf{DeepMind'i StarCraft II stiimulõppe tulemuste reprodutseerimine aktor-kriitik meetoditega}\\
Bakalaureusetöö\\
Roman Ring\\
\end{center}

\noindent 
\textbf{Lühikokkuvõte.} Stiimulõpe on tehisintellekti valdkond, mille uurimisobjektiks on agent, mis navigeerib etteantud keskkonnas eesmärgiga maksimeerida oma tegevustest tulenevat preemiat. Mängud on sobivad keskkonnad stiimulõppe algoritmide testimiseks, kuna nendel on lihtsad reeglid ja selgelt defineeritud preemia. Töö teoreetilises osas uuritakse stiimulõppe populaarsemaid meetodeid, s.h. tehisnärvivõrkude kasutust. Praktilises osas on realiseeritud tehisnärvivõrgul põhinev aktor-kriitik algoritm~\cite{Vinyals2017}. Töös reprodutseeritakse hetke parimaid tulemusi videomängus StarCraft II, mida peetakse tehisintellekti valdkonna järgmiseks verstapostiks pärast male ja Go alistamist. %Praktilises osas on reprodutseeritud DeepMind'i tehisnärvivõrgupõhine arhitektuur ja aktor-kriitik algoritm eesmärgiga lahendada ülesanded tänapäevases mängukeskkonnas StarCraft II.
\\\\
\textbf{Märksõnad:} stiimulõpe, tehisnärvivõrgud
\\\\
\textbf{CERCS teaduseriala:} P176 Tehisintellekt

\pagebreak